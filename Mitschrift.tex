% !TeX spellcheck = de_DE
\documentclass[a4paper,12pt,numbers=noenddot,parskip=full]{scrartcl}
 
\usepackage[utf8]{inputenc}
\usepackage[T1]{fontenc}
\usepackage{lmodern}
\usepackage[ngerman]{babel}
\usepackage{color}
\usepackage[hidelinks]{hyperref}
\usepackage{amsmath,amssymb,amstext,mathtools,amsthm}
\usepackage{subcaption}
\usepackage{float}
\usepackage{wasysym}
\usepackage{stmaryrd}
\usepackage{bbm}
\hypersetup{bookmarksnumbered}



\usepackage{tikz}
\usetikzlibrary{positioning}
\usetikzlibrary{arrows}

\usepackage{enumitem}
\setenumerate{label=\arabic*)}

\newcommand{\setN}{\mathbb{N}}
\newcommand{\setZ}{\mathbb{Z}}
\newcommand{\setQ}{\mathbb{Q}}
\newcommand{\setR}{\mathbb{R}}
\newcommand{\setC}{\mathbb{C}}
\newcommand{\setH}{\mathbb{H}}
\newcommand{\setk}{\Bbbk}
\newcommand{\ldot}{\,.\,}
\newcommand{\Forall}{~\forall}
\newcommand{\Exists}{~\exists}

\newcommand{\scrL}{\mathcal{L}}
\newcommand{\scrA}{\mathcal{A}}
\newcommand{\scrB}{\mathcal{B}}
\newcommand{\scrC}{\mathcal{C}}
\newcommand{\scrD}{\mathcal{D}}
\newcommand{\scrU}{\mathcal{U}}
\newcommand{\scrP}{\mathcal{P}}

\DeclareMathOperator{\Diag}{Diag}

\newcommand{\atdig}{\Diag^{\text{at}}}
\newcommand{\vdig}{\Diag}

\newcommand{\abs}[1]{{\left| #1 \right|}}
\newcommand{\dabs}[1]{{\left\lVert #1 \right\rVert}}
\newcommand{\heading}{\underline}
 
\DeclareMathOperator{\charac}{char}
\DeclareMathOperator{\im}{Im}
\DeclareMathOperator{\dom}{Dom}
\DeclareMathOperator{\dlo}{DLO}
\DeclareMathOperator{\acf}{ACF}
\DeclareMathOperator{\quo}{Quot}
\DeclareMathOperator{\ring}{Ring}


\newtheoremstyle{dotless}{}{}{\normalfont}{}{\bfseries}{}{\newline}{}
	
\theoremstyle{dotless}
\newtheorem{theorem}{Satz}[section]
\newtheorem{corollary}[theorem]{Folgerung}
\newtheorem{proposition}[theorem]{Proposition}
\newtheorem{lemma}[theorem]{Lemma}
\newtheorem{definition}[theorem]{Definition}
\newtheorem{example}[theorem]{Beispiel}
\newtheorem*{axiom}{Axiom}

\makeatletter
\g@addto@macro\th@remark{\thm@headpunct{:}}
\makeatother 


\theoremstyle{remark}
\newtheorem*{remark}{Bemerkung}
 
\hfuzz=5pt 
 
\title{Modelltheorie}

\subtitle{Wintersemester 2019/20 \\ Mitschrift von Floris Remmert}
\author{Prof. Dr. Amador Martin-Pizarro\\Abteilung für mathematische Logik\\Mathematisches Institut\\Albert-Ludwigs-Universität Freiburg}
\date{\today}
 
\begin{document}
	\pagestyle{headings}
\begin{titlepage}
	\maketitle	
	\thispagestyle{empty}
\end{titlepage}
\newpage 
\thispagestyle{empty}
\quad 
\newpage
\tableofcontents 
\thispagestyle{empty}

\newpage
\setcounter{page}{1}
%Sitzung 1
Ziel dieser Vorlesung ist es, eine Aussage der folgenden Qualit"at zu erhalten:
\begin{theorem}[Morley]
	Sei $T$ eine Theorie, welche ein einziges (bis auf Isomorphie) Modell der M"achtigkeit $\aleph_0$ besitzt. Dann besitzt $T$ f"ur jede Kardinalzahl $\kappa > \aleph_0$ ein einziges Modell der M"achtigkeit $\kappa$ (bis auf Isomorphie).
\end{theorem}

\section{Erinnerung}
\begin{definition}
	\begin{itemize}
		\item Eine Sprache $\scrL$ ist eine Kollektion von Konstanten-, Funktions-, und Relationszeichen
		\item Eine $\scrL$-Struktur $\scrA$ besteht aus einer \underline{nicht-leeren} Grundmenge (oder Universum) $A$ zusammen mit Interpretationen der Symbole aus $\scrL$:
		\begin{itemize}
			\item F"ur jedes Funktionszeichen $f$ der Stelligkeit $n$ 
			\begin{equation*}
				f^\scrA : A^n \longrightarrow A
			\end{equation*}
			\item F"ur jedes Relationszeichen $R$ der Stelligkeit $m$
			\begin{equation*}
				R^\scrA \subset A^m
			\end{equation*}
		\end{itemize}
	\item Eine Einbettung $F$ von $\scrA$ nach $\scrB$ ist eine \underline{injektive} Abbildung $F: A \longrightarrow B$, welche mit den Interpretationen kompatibel\footnote{das bedeutet, dass Funktions- und Relationszeichen bei Hin- und R"uckrichtung erhalten bleiben} ist
	\item Ein Isomorphismus ist eine surjektive Einbettung.
	\item $\scrA$ ist eine Unterstruktur von $\scrB$, falls $A \subset B$ und die Inklusion $\iota : A \longrightarrow B$ eine Einbettung bestimmt
	\end{itemize}
\end{definition}

\begin{remark}
	Sei $\scrB$ eine $\scrL$-Struktur, $\emptyset \ne A \subset B$. Dann gibt es eine Unterstruktur von $\scrB$, welche von A erzeugt wird.
	
	Das Universum besteht aus $A$ zusammen mit dem Abschluss von $A$ unter allen Interpretationen der Funktionszeichen von $\scrL$.
\end{remark}

\begin{definition}
	Sei $(I, <)$ eine partielle Ordnung. Die Ordnung ist \underline{gerichtet}, falls f"ur $i, j \in I$ gibt es $k \in I$ mit $i \leq k$ und $j \leq k$.
\end{definition}

\begin{remark}
	Sei $(\scrA_i)_{i \in I}$ eine Familie von $\scrL$-Strukturen indexiert nach der gerichteten partiellen Ordnung $I$ derart, dass f"ur $i \leq j$ gilt: $\scrA_i \underset{US}{\subset} \scrA_j$.
	
	Die Menge $A=\underset{i \in I}{\bigcup} A_i$ ist das Universum einer (eindeutig bestimmten) $\scrL$-Struktur
	\begin{equation}
		\scrA = \underset{i \in I}{\bigcup} \scrA_i \label{1:str}
	\end{equation}
	
	Falls $I$ eine lineare Ordnung ist, dann ist $(\scrA_i)_{i \in I}$ eine \underline{Kette}.
	
	\underline{Zu \ref{1:str}:} \begin{itemize}
		\item $c^\scrA=c^{\scrA_i}$ f"ur ein (alle) $i \in I$, denn $c^{\scrA_i}=c^{\scrA_j}=c^{\scrA_k}$, {wegen gerichteter Ordnung}
		\item $a_1, \dots a_n \in A = \underset{i \in I}{\bigcup} A_i \Longrightarrow \exists i \in I$ mit $a_1, \dots, a_n \in A_i$.
		Also ist $f^\scrA (a_1, \dots, a_n) = f^{\scrA_i} (a_1, \dots, a_n)$ wohldefiniert.
		\item $(a_1, \dots, a_m) \in R^\scrA$ genau dann, wenn es ein $i \in I$ gibt mit $a_1, \dots, a_m \in A_i$ und $(a_1, \dots, a_m) \in R^{\scrA_i}$
	\end{itemize}

	\underline{Beachte}, dass $\scrA_i \underset{US}{\subset} \scrA$ f"ur alle $i \in I$.
\end{remark}

\begin{definition}
	Eine atomare Formel ist ein Ausdruck der Form $(t_1 \dot= t_2)$, $t_1, \dots, t_k$ Terme, $R(t_1, \dots, t_k)$.
	
	Die Kollektion von Formeln ist die kleinste Klasse, welche alle atomaren Formeln enth"alt und derart, dass:
	\begin{align*}
		\varphi \text{ Formel} &\Longrightarrow \lnot \varphi \text{ Formel}\\
		\varphi, \psi \text{ Formel} &\Longrightarrow (\varphi \lor \psi) \text{ Formel}\\
		\varphi \text{ Formel}, x \text{ Variable} &\Longrightarrow \exists x \varphi \text{ Formel, ($x$ hei"st dann "`gebunden"')}
	\end{align*}
	
	\underline{Abk.:} \begin{align*}
		&(\varphi \land \psi) &&= \lnot(\lnot\varphi\lor\lnot\psi)\\
		&\forall x \varphi &&= \lnot \exists x \lnot \varphi\\
		&(\varphi \rightarrow \psi) &&= (\lnot \varphi \lor \psi)\\
		&(\varphi \leftrightarrow \psi) &&= ((\varphi \rightarrow \psi) \land (\psi \rightarrow \varphi))
	\end{align*}
\end{definition}

\begin{remark}
	
	\begin{itemize}
		\item Jede Formel $\varphi [x_1, \dots, x_n]$ l"asst sich in \underline{pr"anexer Normalform} umschreiben:	
		$Q_1 y_1 Q_2 y_2 \dots Q_m y_m \psi [x_1, \dots, x_n, y_1, \dots, y_m]$, mit $Q_i \in \{\forall, \exists\}$. Das ist eine quantorfreie Formel, diese l"asst sich weiter zerlegen in KNF bzw. DNF.
		\item Eine Formel ohne freie Variablen ist eine Aussage
		\item Eine Theorie ist eine Kollektion von Aussagen
	\end{itemize}
\end{remark}

\begin{example}
	Sei $\scrA$ eine $\scrL$-Struktur. Erweitere die Sprache zu der Sprache $\scrL_A = \scrL \cup \{d_a\}_{a \in A}.$
	
	$\scrA$ ist eine $\scrL_A$-Struktur, $d_a^\scrA = a$.
	\begin{itemize}
		\item $\atdig(\scrA) = \{$quantorenfreie $\scrL_A$-Aussagen $\chi$ mit $\scrA \models \chi\}$ hei"st "`atomares Diagramm"'
		\item $\vdig (\scrA) = \{\scrL$-Aussagen $\theta$ mit $\scrA \models \theta\}$ hei"st "`vollst"andiges Diagramm"'
	\end{itemize}
	Sei nun $\scrB$ eine $\scrL_A$-Struktur.
	\begin{align*}
		\scrB \models \atdig(\scrA) \Leftrightarrow &\scrA \hookrightarrow \scrB \text{ einbetten l"asst}\\
		&A \longrightarrow B\\
		&a \mapsto d_a^\scrB\\
		\scrB \models \vdig(\scrA) \Leftrightarrow &\text{ die obige Abbildung ist \underline{elementar}}\\
		\scrA \models \varphi[a_1, \dots, a_n] \Leftrightarrow &\scrB \models \varphi [F(a_1), \dots, F(a_n)], a_1, \dots a_n \in A, \varphi[x_1, \dots, x_n] \text{ Formel}
	\end{align*}
\end{example}

\begin{definition}
	\begin{itemize}
		\item $T$ ist konsistent, falls $T$ ein Modell besitzt.
		\item $T$ ist vollst"andig, falls  $T$ konsistent ist und je zwei Modelle von $T$ elementar "aquivalent sind.
	\end{itemize}
\end{definition}

\begin{theorem}[Kompaktheitssatz]
	Eine Theorie ist genau dann konsistent, wenn sie \underline{endlich konsistent}\footnote{endlich konsistent bedeutet: jede endliche Teilmenge der Theorie besitzt ein Modell.} ist.
\end{theorem}

\underline{Wie zeigen wir, dass $\scrA \equiv \scrB$?}
\begin{theorem}[Back \& Forth]
	$S = \{F: \underset{\underset{US}{\subset} \scrA}{\scrC} \longrightarrow \underset{\underset{US}{\subset} \scrB}{\scrD}, F$ partieller Isomorphismus zwischen $\scrC$ und $\scrD$ geeignet\footnote{bspw. endlich erzeugt}$\}$.
	
	\underline{Back:} F"ur alle $F \in S$ und $b \in B$, $F:\scrC \longrightarrow \scrD$ gibt es $G \in S$ mit $G \supset F$ Erweiterung und $b \in \im (G)$.
	
	\underline{Forth:} F"ur alle $F \in S$ und $a \in A$, $F: \scrC \longrightarrow \scrD$ gibt es $H \in S$, mit $H \supset F$ Erweiterung mit $a \in \dom (H)$
	
	$\scrA$ und $\scrB$ hei"sen dann "`Back \& Forth "aquivalent"'
	
	$\rightarrow$ ist jedes $F \in S$ \underline{elementar}, so gilt insbesondere $\scrA \equiv \scrB$.
\end{theorem}

%Sitzung 2
\part{Theorien und Quantorenelimination}
\section{Tarskis Test}
\begin{lemma}[Tarskis Test]
	Sei $\scrB$ eine $\scrL$-Struktur und $A \subset B$ Teilmenge derart, dass f"ur jede $\scrL$-Formel $\varphi[x_1, \dots, x_n]$ und Elemente $a_1, \dots, a_n \in A$:\\
	\underline{falls}:
	\begin{equation}\label{tarski:bed}
		\scrB \models \varphi [a_1, \dots, a_n, b] \text{ f"ur ein } b \in B \Rightarrow \text{ existiert } a \in A \text{ sodass }\scrB \models \varphi[a_1, \dots, a_n, a]
	\end{equation}
	\underline{dann} ist $A$ das Universum einer elementaren Unterstruktur von $\scrB$.
	
	Insbesondere: Falls $\scrA \subset \scrB$ Unterstruktur, ist $\scrA \preceq \scrB \Leftrightarrow A$ erf"ullt \ref{tarski:bed}.
\end{lemma}
\begin{proof}
	\underline{Betrachte} $A \neq \emptyset \rightarrow$ Betrachte $\varphi[y] = (y \dot= y)$. $B \neq \emptyset \rightarrow \exists b \in B$ mit $\scrB \models \varphi[b]$. $\hookrightarrow \exists a \in A$ mit $\scrB \models \varphi[a]$
	
	\underline{Beh.:} F"ur jedes Konstantenzeichen $c \in \scrL$ ist $c^\scrB \in A$. $\hookrightarrow \varphi[y] = (y \dot= c)$, $\scrB \models \varphi[c^\scrB] \Rightarrow$ es gibt $a \in A$ mit $a=c^\scrB$.
	
	\underline{Beh.:} $A$ ist unter den Funktionen $f^\scrB$ abgeschlossen, f"ur jedes Funktionszeichen $f \in \scrL$.
	
	Sei $\varphi[x_1, \dots, x_n, y] = (y \dot= f(x_1, \dots, x_n))$ $\checkmark$
	
	F"ur $R \in \scrL$ $m$-stellig setze $R^\scrA = A^m \cap R^\scrB$ $\longrightarrow$ somit bildet $A$ eine $\scrL$-Unterstruktur $\scrA$ von $\scrB$.
	
	Noch zu zeigen: $\scrA \preceq \scrB$, d. h. $\varphi[x_1, \dots, x_n]$ $\scrL$-Formel.
	
	Seien dazu $a_1, \dots, a_n \in A$.
	\begin{equation}\label{tarksi:bew}
		\scrA \models \varphi[a_1, \dots, a_n] \Leftrightarrow \scrB \models \varphi[a_1, \dots,a_n]
	\end{equation}
	
	Induktiv "uber den Aufbau von $\varphi$.
	
	$\varphi$ ist atomar $\longrightarrow$ $\checkmark$
	\begin{align*}
		\scrA &\not\models \psi[a_1, \dots, a_n] \Leftrightarrow &\scrB &\not\models \psi[a_1, \dots, a_n]\\
		&\Updownarrow &&\Updownarrow\\
		\scrA &\models \varphi[a1, \dots, a_n] &\scrB &\models \phi[a_1, \dots, a_n]
	\end{align*}
	
	$\varphi= \lnot \psi \longrightarrow$ $\checkmark$
	
	$\varphi = (\psi_1 \lor \psi_2) \longrightarrow \checkmark$
	
	$\varphi = \exists y \psi[x_1, \dots, x_n, y]$: $\scrA \models \varphi[a_1, \dots, a_n] \Rightarrow$ es gibt ein $a \in A$ sodass $\scrA \models \psi[a_1, \dots, a_n, a]\\ {\underset{\ref{tarksi:bew}}{\Rightarrow} \scrB \models \psi[a_1, \dots, a_n, a]} \text{ f"ur ein } a \in A \subset B \Rightarrow \scrB \models \varphi[a_1, \dots, a_n]$
	
	$\scrB \models \varphi[a_1, \dots, a_n] \Rightarrow$ es gibt $b \in B$ mit $\scrB \models \psi[a_1, \dots, a_n, b] \underset{\ref{tarski:bed}}{\Rightarrow}$ es gibt ein $a \in A$ mit $\scrB \models \psi[a_1, \dots, a_n, a] \underset{\ref{tarksi:bew}}{\Rightarrow} \scrA \models \psi[a_1, \dots, a_n, a] \Rightarrow \scrA \models \varphi[a_1, \dots, a_n]$.
	
	\underline{F"ur "`insbesondere"':} Angenommen, dass $\scrA \preceq \scrB$. Sei $\varphi[x_1, \dots, x_n, y]$ eine $\scrL$-Formel, $a_1, \dots, a_n \in A$. Dann: $\scrB \models \varphi[a_1, \dots, a_n, b]$ f"ur ein $b \in B$ $\Rightarrow \scrB \models (\exists y \varphi)[a_1, \dots, a_n] \\
	\underset{\scrA \preceq \scrB}{\Rightarrow} \scrA \models (\exists y \varphi)[a_1, \dots, a_n] \Rightarrow$ es gibt ein $a \in A$ mit $\scrA \models \varphi[a_1, \dots, a_n, a] {\underset{\scrA \preceq \scrB}{\Rightarrow} \scrB \models \varphi[a_1, \dots, a_n, a]} \checkmark$
\end{proof}

\begin{proposition}[aufw"arts L"owenheim-Skolem]
	Sei $\scrA$ eine unendliche $\scrL$-Struktur, und $\kappa < \max\{|A|,|\scrL|\}$. Dann gibt es eine elementare $\scrL$-Erweiterung $\scrB \geq \scrA$ der M"achtigkeit $\kappa$.
\end{proposition}
\begin{proof}
	$\vdig(\scrA) \cup \{\lnot(c_\alpha \dot= c_\beta)\}_{\alpha \neq \beta < \kappa}$, wobei $\{c_\alpha\}_{\alpha<\kappa}$ eine Menge neuer Konstantenzeichen ist, ist konsistent weil sie endlich konsistent\footnote{Kompaktheit} ist.
	
	Aus der Konstruktion von Henkin hat $\vdig(\scrA) \cup \{\lnot(c_\alpha \dot= c_\beta)\}_{\alpha \neq \beta < \kappa}$ ein Modell der M"achtigkeit der Sprache.
	
	$\rightarrow$ ein Modell der \underline{M"achtigkeit $\kappa$}
\end{proof}

\begin{remark}
	$|A| = n \in \setN$, $\scrB \succeq \scrA \Rightarrow |B| = n$
\end{remark}

\begin{proposition}[abw"arts L"owenheim-Skolem]\label{Low:ab}
	Sei $\scrB$ eine $\scrL$-Struktur und $S \subset B$ beliebig. Dann gibt es eine elementare Unterstruktur $\scrA \preceq \scrB$ mit $A \supset S$ und $|A|\leq \max\{|S|,|\scrL|,\aleph_0\}$.
\end{proposition}

\begin{remark}
	$\setC$ in der Ringsprache $\scrL_\text{Ring}$, $S=\emptyset \Rightarrow$ es gibt eine abz"ahlbare elementare Unterstruktur von $\setC$. $\rightarrow \overline{\setQ} \preceq \setC$.
\end{remark}

\begin{proof}[Beweis \ref{Low:ab}]
	Setze $S_0 = S$. Angenommen $S_k$ wurde bereits konstruiert, w"ahle f"ur jedes $n \in \setN$, jede $\scrL$-Formel $\varphi[x_1, \dots, x_n, y]$ und Elemente $a_1, \dots, a_n \in S_k$ ein Element $a_{\varphi[a_1, \dots, a_n, y]} \in B$ derart, dass $\scrB \models ((\exists y \in \varphi)[a_1, \dots, a_n] \rightarrow \varphi[a_1, \dots, a_n, a_{\varphi[a_1, \dots, a_n, y]}])$. Setze $S_{k+1} = S_k \cup \{a_\varphi\}_{\varphi \scrL \text{-Formel, }(a_1, \dots, a_n)\in S_k}$
	
	Definiere $A = \underset{k \in \setN}{\bigcup} S_k \supset S$. Wir "uberpr"ufen, dass $A$ den Test von Tarski erf"ullt. Sei $\varphi = \varphi[x_1, \dots, x_n, y]$ eine $\scrL$-Formel, $a_1, \dots, a_n \in A$.
	
	$\scrB \models \varphi[a_1, \dots, a_n, b]$ f"ur ein $b \in B \Rightarrow$ es gibt ein $k \in \setN$ mit $a_1, \dots a_n \in S_k \Rightarrow$ es gibt ein $a_{\varphi[a_1, \dots, a_n, y]} \in S_{k+1} \subset A$ mit $\scrB \models \varphi[a_1, \dots, a_n, a] \checkmark$
	
	Ferner ist $|A| \leq \max \{\aleph_0, |\scrL|, |S|\}$.
\end{proof}

\begin{corollary}
	Sei $(\scrA_i)_{i \in I}$ eine gerichtete Familie von $\scrL$-Strukturen, sodass f"ur $i \leq j$ ist $\scrA_i \preceq \scrA_j$. Dann ist $\scrA = \underset{i \in I}{\bigcup} \scrA_i$ eine elementare Erweiterung jeder $\scrA_i$.
\end{corollary}
\begin{proof}
	Wir beweisen induktiv "uber den Aufbau von $\varphi=\varphi[x_1, \dots, x_n]$, dass f"ur alle $i \in I$, f"ur alle $a_1, \dots, a_n \in A_i$: $\scrA_i \models \varphi[a_1, \dots, a_n] \Leftrightarrow \scrA \models \varphi[a_1, \dots, a_n]$.
	
	$\varphi$ atomar $\rightarrow$ klar, denn $\scrA_i \underset{US}{\subset} \scrA$
	
	$\varphi = \lnot \varphi \Rightarrow$ ok!
	
	$\varphi = (\varphi_1 \lor \varphi_2) \Rightarrow$ ok!
	
	$\varphi = \exists y \psi[x_1, \dots, x_n, y]$: $\scrA_i\models \varphi[a_1, \dots, a_n] \Rightarrow$ es gibt ein $a \in A_i$ mit $\scrA_i\models \psi[a_1, \dots,a_n, a] \\
	\underset{\text{ind. "uber }\psi}{\Rightarrow} \scrA \models \psi[a_1, \dots, a_n, a] \Rightarrow \scrA \models \varphi[a_1, \dots, a_n]$
	
	$\scrA \models \varphi[a_1, \dots, a_n] \Rightarrow$ es gibt ein $b \in A = \underset{i \in I}{\bigcup} A_i$ mit $\scrA \models \psi[a_1, \dots, a_n, b] \Rightarrow$ es gibt $j \in I$ mit $b \in A_j \Rightarrow$ es existiert $k \in I$ mit $i \leq k$, $j \leq k$, $a_1, \dots, a_n, b \in A_k \\
	\Rightarrow \scrA_k \models \psi[a_1, \dots, a_n, b] \underset{\scrA_i \preceq \scrA_k}{\Rightarrow}$ es gibt ein $a \in A_k$ mit $\scrA_i \models \psi[a_1, \dots, a_n, a] \Rightarrow {\scrA_i \models \varphi[a_1, \dots, a_n]}$.
\end{proof}

\section{Quantorenelimination}
\begin{definition}
	Eine Theorie $T$ hat Quantorenelimination, falls jede $\scrL$-Formel $\varphi[x_1, \dots, x_n]$ "aquivalent modulo $T$ zu einer quantorenfreien $\scrL$-Formel $\psi[x_1, \dots, x_n]$ ist.
	\begin{equation*}
		T \models \forall x_1 \dots \forall x_n (\varphi[x_1, \dots, x_n] \leftrightarrow \psi[x_1, \dots, x_n])
	\end{equation*}
\end{definition}

\begin{example}
	Sei $\scrL \coloneqq (\setR, 0, 1, +, -, \cdot)$ gegeben. Betrachte die Menge $\{(a,b,c) \in \setR^3 | a \neq 0$ und es gibt $x \in \setR$ mit $ax^2+bx+c=0\} = \{(a,b,c) \in \setR^3 | a \neq 0$ und $b^2-4ac \geq 0\}$.
	
	Diese Formel ist in $\scrL$ nicht "aquivalent zu einer quantorenfreien Formel, in ${\scrL_1 \coloneqq (\setR, 0, 1, +, -, \cdot, <)}$ hingegen doch. Somit ist die Menge in $\scrL_1$ quantorenfrei.
\end{example}

%Sitzung 3
\begin{remark}
	\begin{itemize}
		\item Wenn $T$ inkonsistent ist, dann hat $T$ immer Quantorenelimination
		\item Wenn $T$ Quantorenelimination hat, und $\scrA , \scrB \models T$ mit $\scrA \underset{\text{US}}{\subset} \scrB \Rightarrow \scrA \preceq \scrB$ \marginpar{"Ubung}
	\end{itemize}
\end{remark}

\begin{definition}
	\begin{itemize}
		\item Eine einfache Existenzformel ist eine Formel der Form $\varphi[x_1, \dots, x_n]=\exists y \psi[x_1, \dots, x_n, y]$
		\item Eine primitive Existenzformel ist eine Formel der Form $\varphi[x_1, \dots, x_n]=\psi[x_1, \dots, x_n,y]$, wobei $\psi$ eine endliche Konjunktion von atomaren Formeln und Negationen ist
	\end{itemize}
\end{definition}

\begin{lemma}
	Eine (konsistente) Theorie T hat genau dann Quantorenelimination, wenn jede primitive Existenzformel zu einer quantorenfreien Formel "aquivalent modulo T ist.
\end{lemma}
\begin{proof}
	"`$\Rightarrow$"': klar
	
	"`$\Leftarrow$"': Beachte, $\exists y (\psi_1 \lor \psi_2) \leftrightarrow (\exists y \psi_1 \lor \exists y \psi_2)$. Insbesondere, wenn $T$ Quantorenelimination f"ur primitive Existenzformeln hat, dann hat T Quantorenelimination f"ur einfache Existenzformeln.
	\begin{equation*}
		\underset{\text{einfache Existenzformel}}{\varphi} = \exists y \underbrace{\psi[x_1, \dots, x_n]}_{\text{umschreiben in DNF}} \sim \exists y (\psi_1 \lor \dots \lor \psi_n) \sim \underbrace{\bigvee\limits^n_{i=1} \exists y \psi_i}_{\text{primitive Existenzformel}}
	\end{equation*}
	
	Zu zeigen: Jede beliebige Formel $\varphi[x_1, \dots, x_n]$ ist "aquivalent zu einer quantorenfreien Formel modulo $T$.
	\begin{equation*}
		\varphi[x_1, \dots, x_n] \underbrace{\sim}_{\substack{\text{pr"anexe}\\ \text{Normalform}}} Q_1 y_1 \dots Q_m y_m \underbrace{\psi[x_1, \dots, x_n, y_1, \dots, y_m]}_{\text{quantorenfrei}}, \text{ wobei }Q_i \in \{\forall, \exists\}
	\end{equation*}
	
	Induktion "uber $m$:
	\begin{itemize}
		\item[$m=0$:] $\checkmark$
		\item[$m=1$:] $\varphi = Q \underbrace{\psi[x_1, \dots, x_n, y]}_{\text{quantorenfrei}}$
		\begin{itemize}
			\item[$Q=\exists$] $\varphi$ einfache Existenzformel $\checkmark$
			\item[$Q=\forall$] $\varphi \sim \lnot \underbrace{\exists y \lnot \psi}_{\substack{\text{einfache}\\\text{Existenzformel }} \rightarrow \text{ eliminieren }\rightarrow \checkmark}$
		\end{itemize}
		\item[$m-1\rightarrow m$:] $\varphi[x_1, \dots, x_n]=Q_1 y_1 Q_2 y_2 \dots \underbrace{Q_m y_m \psi[x_1, \dots, x_n, y_1, \dots, y_m]}_{\varphi' [x_1, \dots, x_n, y_1, \dots, y_{m-1}]}$. $\varphi'$ ist eine einfache Existenzformel, wir eliminieren also:
		
		$\underbrace{\hfil}_{m-1 \text{ viele Quantoren}} \underbrace{\Theta[x_1, \dots, x_n, y_1, \dots, y_{m-1}]}_{\text{quantorenfrei}}$
		
		$\Rightarrow$ Induktion
	\end{itemize}

\end{proof}

\begin{example}
	Sei $\mathcal{K} = \{\text{unendliche Mengen}\}$. Diese Klasse l"asst sich definieren durch die Theorie \marginpar{$\exists^\infty x$}$T=\{\exists x_1 \dots \exists x_n(\bigwedge\limits^n_{i \neq j =1}\lnot(x_i \dot= x_j))\}_{n \in \setN}$.
	Diese Theorie ist vollst"andig!
	Betrachte jetzt die definierbaren Mengen:
	\begin{equation*}
		\{b \in A | \scrA \models \underbrace{\underbrace{\varphi}_{\text{quantorenfrei}}[b, a_1, \dots, a_m]}_{\underset{\text{endlich oder koendlich}}{\updownarrow}} \}
	\end{equation*}
\end{example}

\begin{lemma}[Trennungslemma]
	Seien $T_1$ und $T_2$ zwei $\scrL$-Theorien, und $\Delta$ eine Kollektion von $\scrL$-Aussagen, welche unter endlichen Konjunktionen und Disjunktionen abgeschlossen ist. Folgende Eigenschaften sind "aquivalent:
	\begin{enumerate}
		\item Es gibt eine Aussage $\chi \in \Delta$ mit $T_1 \models \chi$
		\item F"ur alle $\scrA \models T_1$, $\scrB \models T_2$ gibt es eine Aussage $\chi \in \Delta$ mit $ \scrA \models \chi, \scrB \models \lnot \chi$
	\end{enumerate}
\end{lemma}

\begin{remark}
	Das ganze ist trivial f"ur inkonsistente Theorien.
\end{remark}

\begin{proof}
	\underline{$1\Rightarrow2$:} trivial!
	
	\underline{$2 \Rightarrow 1$:} OBdA $T_1, T_2$ konsistent. Sei $\scrA \models T_1$, setze ${\Sigma_\scrA =\{\chi, \chi \text{ Aussagen in } \Delta\text{ mit } \scrA \models \chi \}}$. 
	
	Betrachte jetzt $T_2 \cup \Sigma_\scrA$. Ist diese Theorie konsistent? Nein: W"are $\scrB \models T_2 \cup \Sigma_\scrA \hookrightarrow$ es gibt $\chi \in \Delta$ mit $\scrA \models \chi, \scrB \models \lnot \chi \Rightarrow \chi \in \Sigma_\scrA \Rightarrow \scrB \models \chi$. Widerspruch!
	
	Das bedeutet (wegen Kompaktheit), dass es $\chi_1, \dots, \chi_r \in \Sigma_\scrA$ gibt mit $T_2 \cup \{\chi_1, \dots, \chi_r\}$ inkonsistent.
	\begin{equation*}
		\hookrightarrow T_2 \models \bigvee\limits_{i=1}^{r} \lnot \chi_i \Rightarrow T_2 \models \lnot (\underbrace{\bigwedge\limits_{i=1}^{r} \chi_i}_{=\chi_\scrA \in \Delta })
	\end{equation*}
	Das hei"st f"ur jedes $\scrA \models T_1$ gibt es $\chi_\scrA \in \Delta$ mit $T_2 \models \lnot \chi_\scrA$ und $\scrA \models \chi_\scrA$.
	
	Sei nun $T_1 \cup \{\lnot \chi_\scrA \}_{\scrA \models T_1}$.\footnote{Ist das "uberhaupt eine Menge? Es gen"ugt die Einschr"ankung bis auf Isomorphie, das sollte reichen\dots} $\hookrightarrow$ inkonsistent nach Konstruktion.
	
	$\underset{\text{Kompaktheit}}{\Rightarrow}$ es existieren $\chi_{\scrA_1}, \dots \chi_{\scrA_n}$ mit $T_1 \cup \{\lnot \chi_{\scrA_1}, \dots, \chi_{\scrA_n} \}$ inkonsistent. Also:\\ 
	$T_1 \models \bigvee\limits_{j=1}^n \chi_{\scrA_j} \eqqcolon \chi \in \Delta$
	
	$T_1 \models \chi$. Wollen zeigen: $T_2 \models \lnot \chi$. Aber $T_2 \models \lnot \chi_{\scrA_i}, 1 \leq i \leq n$.
\end{proof}

\begin{corollary}
	Zwei Theorien $T_1$ und $T_2$ werden von einer quantorenfreien Aussage getrennt, wenn je zwei Modelle $\scrA \models T_1$ und $\scrB \models T_2$ von einer quantorenfreien Aussage getrennt werden.
	\begin{equation*}
		\rightarrow \exists\chi \text{ quantorenfrei }: \scrA \models \chi \text{ und } \scrB \models \lnot \chi
	\end{equation*}
\end{corollary}


%Sitzung 4
\begin{theorem}
	Sei $T$ eine Theorie. Folgende Aussagen sind "aquivalent:
	\begin{enumerate}
		\item $T$ hat Quantorenelimination. \label{beh:1}
		\item \label{beh:2}Gegeben Modelle $\scrA, \scrB \models T$ und endlich erzeugte Unterstrukturen $\langle c_1, \dots, c_n \rangle_\scrA = \scrC \subset \scrA$, $\langle d_1, \dots, d_n \rangle_\scrB = \scrD \subset \scrB$, wobei $\scrC \simeq \scrD$ und $\varphi[x_1, \dots, x_n]$ eine Formel. Dann gilt:
		\begin{equation*}
			\scrA \models \varphi[c_1, \dots, c_n] \Rightarrow\footnote{Durch vertauschen von $\scrA$ und $\scrB$ gilt hier sogar $\Leftrightarrow$.} \scrB \models \varphi[d_1, \dots, d_n]
		\end{equation*}
		\item \label{beh:3} Gegeben Modelle $\scrA, \scrB$ mit isomorph erzeugten Unterstrukturen $\langle c_1, \dots, c_n \rangle_\scrA = \scrC \simeq \scrD = \langle d_1, \dots, d_n \rangle_\scrB$ wie in \ref{beh:2} und f"ur alle $\varphi[x_1, \dots, x_n]$ primitive Existenzformel, gilt:
		\begin{equation*}
			\scrA \models \varphi[c_1, \dots, c_n] \Rightarrow \scrB \models \varphi[d_1, \dots, d_n]
		\end{equation*}
	\end{enumerate}
	Ferner, falls $T$ konsistent ist, \ref{beh:1} gilt und je zwei Modelle von $T$ isomorphe endlich erzeugte Unterstrukturen besitzen, dann ist $T$ vollst"andig mit Quantorenelimination.
\end{theorem}

\begin{remark}
	Wie benutzen wir diesen Satz? %TODO Grafik erstellen!
	Letztlich wollen wir Back-\&-Forth-"Aquivalenz zeigen.
\end{remark}

\begin{proof}
	\underline{$\ref{beh:1}\Rightarrow\ref{beh:2}$:} Sei $\varphi[x_1, \dots, x_n]$. T hat Quantorenelimination $\longleftarrow$ es gibt $\psi[x_1, \dots, x_n]$ quantorenfrei mit: $T \models \forall \vec{x} (\varphi[\vec{x}] \leftrightarrow \psi[\vec{x}])$
	\begin{align*}
		&&\scrA \models \varphi[c_1, \dots, c_n]\\
		&\underset{\scrA \models T}{\Leftrightarrow} &\scrA \models \psi[c_1, \dots, c_n]\\
		&\underset{\psi \text{ quantorenfrei}}{\Leftrightarrow} &\scrC \models \psi[c_1, \dots, c_n]\\
		&\underset{\scrC \simeq \scrD}{\Leftrightarrow} &\scrD \models \psi[d_1, \dots, d_n]\\
		&\Leftrightarrow &\scrB \models \psi[d_1, \dots, d_n]\\
		&\underset{\scrB \models T}{\Leftrightarrow} &\scrB \models \varphi[d_1, \dots, d_n]
	\end{align*}
	
	\underline{$\ref{beh:2} \Rightarrow \ref{beh:3}$:} klar.
	
	\underline{$\ref{beh:3} \Rightarrow \ref{beh:1}$:} Um zu zeigen, dass $T$ Quantorenelimination besitzt, gen"ugt es nur primitive Existenzformeln $\varphi[x_1, \dots, x_n]$ zu betrachten. 
	
	Seien dazu $e_1, \dots, e_n$ neue Konstantenzeichen. Betrachte die Sprache $\scrL \cup \{e_1, \dots, e_n \}$, sowie die Theorien $T_1 = T \cup \{\varphi[e_1, \dots, e_n] \}$ und $T_2 = T \cup \{\lnot \varphi[e_1, \dots, e_n]\}$.
	
	Falls $T_1$ und $T_2$ durch eine quantorenfreie Aussage $\underbrace{\psi[e_1, \dots, e_n]}_{\substack{\text{quantorenfreie}\\ \scrL \text{-Formel}}}$ in $\scrL \cup \{e_1, \dots, e_n \}$ trennbar sind, so folgt:
	\begin{align*}
		&T \cup \{\varphi[\vec{e}] \} \models \psi[\vec{e}] &\Rightarrow T &\models (\varphi[\vec{e}] \rightarrow \psi[\vec{e}])\\
		&T \cup \{\lnot\varphi[\vec{e}] \} \models \lnot\psi[\vec{e}] &\Rightarrow T &\models (\lnot\varphi[\vec{e}] \rightarrow \psi[\vec{e}])\\
		&\Rightarrow T = (\psi[\vec{e}] \rightarrow \varphi[\vec{e}]) &\underset{\text{Aufgabe\footnotemark}}{\Rightarrow} T &\models \forall \vec{x} (\varphi[\vec{x}] \leftrightarrow \underbrace{\psi[\vec{x}]}_{\text{quantorenfrei}})
	\end{align*}
	\footnotetext{weil $e_1, \dots, e_n$ \underline{neue} Konstantenzeichen sind}
	Sonst, falls also $T_1, T_2$ nicht trennbar sind, gibt es zwei Modelle $\scrA \models T_1 \cup \{\varphi[\vec{e}] \}, \scrB \models T \cup \{\lnot \varphi[\vec{e}] \}$, welche alle quantorenfreien Aussagen in $\scrL \cup \{e_1, \dots, e_n \}$ gleich erf"ullen.
	
	Seien $c_1=e^\scrA_i, d_i=e^\scrB_i$. Betrachte jetzt $\langle c_1, \dots, c_n \rangle_\scrA \underset{\scrL \text{-US}}{\subset} \scrA \mid_\scrL$ und $\langle d_1, \dots, d_n \rangle_\scrB \underset{US}{\subset} \scrB \mid_\scrL$. Es gilt: $\scrA \models \varphi[c_1, \dots, c_n]$ und $\scrB \models \lnot \varphi[d_1, \dots, d_n]$.
	
	Um einen Widerspruch zu bekommen gen"ugt es zu zeigen, dass $\scrC \simeq \scrD, c_i \mapsto d_i$.
	\begin{align*}
		C &\longrightarrow D:\\
		\underbrace{t^\scrA[c_1, \dots, c_n]}_{\scrL \text{-Term}} &\mapsto t^\scrB[d_1, \dots, d_n]
	\end{align*}
	\underline{Ist diese Abbildung wohldefiniert?}
	\begin{align*}
		\text{Angenommen } &t_1^\scrA[c_1, \dots, c_n] = t_2^\scrA[c_1, \dots, c_n]\\
		\Leftrightarrow &\underbrace{\scrA}_{\text{als } \scrL \cup \{e_1, \dots, e_n \} \text{-Struktur}} \models (\underbrace{t_1[e_1, \dots, e_n] \dot= t_2[e_1, \dots, e_n]}_{\text{quantorenfreie Aussage}})\\
		\Leftrightarrow &\scrB \models (t_1[\vec{e}] \dot= t_2[\vec{e}])\\
		\Leftrightarrow &t_1^\scrB[d_1, \dots, d_n] = t_2^\scrB[d_1, \dots, d_n]\\
		\longrightarrow &\text{ wohldefiniert und injektiv}
	\end{align*}
	induktiv "uber den Aufbau zeigen wir: Das ist ein Isomorphismus.
	
	\underline{Zu "`ferner"':} Angenommen $T$ hat Quantorenelimination, ist konsistent und je zwei Modelle $\scrA, \scrB \models T$ haben isomorphe, endlich erzeugte Unterstrukturen
	\begin{equation*}
		\underset{c_i \mapsto d_i}{\langle c_1, \dots, c_n \rangle_\scrA = \overset{\subset \scrA}{\scrC} \simeq \overset{\subset \scrB}{\scrD} = \langle d_1, \dots, d_n \rangle_\scrB}
	\end{equation*}
	$T$ ist vollst"andig $\Leftrightarrow \scrA \equiv \scrB$.
	Sei $\chi$ eine $\scrL$-Aussage und schreibe $\chi = \chi[x_1, \dots, x_n]$.
	\begin{equation*}
		\scrA \models \chi \Leftrightarrow \scrA \models \chi[c_1, \dots, c_n] \underset{\ref{beh:2}}{\Leftrightarrow} \scrB \models \chi[d_1, \dots, d_n] \Leftrightarrow \scrB \models \chi
	\end{equation*}
\end{proof}

\section{Beispiele klassischer Theorien}
\begin{example}
	$T=\exists^\infty$ hat Quantorenelimination und ist vollst"andig.
\end{example}

\begin{example}
		$\dlo$ (dichte lineare Ordnung ohne Randpunkte). Sei $\scrL = \{<\}$.
	\begin{align*}
		\dlo = &\{\forall x (\lnot x<x) \}\\
		&\cup \{\forall x \forall y \forall z ((x<y \land y<z) \rightarrow (x<z)) \}\\
		&\cup \{\forall x \forall y ((x=y)\lor(x<y)\lor(y<x)) \}\\
		&\cup \{\forall x \forall y \exists z ((x<y)\rightarrow(x<z<y)) \}\\
		&\cup \{\forall x \exists u \exists v (u<x<v) \}\\
		&\cup \{\exists x (x=x) \}
	\end{align*}
	Diese Theorie ist vollst"andig und hat Quantorenelimination.
	Es gibt zwei Methoden, um Quantorenelimination zu zeigen:
	\begin{enumerate}
		\item
		\begin{align*}
			&\varphi[x_1, \dots,x_n] &= \exists y (\bigwedge\limits_i \overbrace{\Theta_i[x_1, \dots,x_n, y]}^{\substack{\text{atomar oder} \\ \text{Negation davon}}})\\
			&&= \exists y (\psi_1[x_1, \dots, x_n] \land \bigwedge\limits_i \substack{x_i = y \\ x_i \neq y \\ x_i < y \\ y < x_i})\\ \\
			&x_i=y \land x_j=y \Leftrightarrow x_i = x_j\\
			&x_i = y \land y<x_j \Leftrightarrow x_i < x_j &\longrightarrow \text{ induktiv lassen sich alle Quantoren eliminieren}
		\end{align*}
		\item Gegeben $\langle c_1, \dots, c_n \rangle_\scrA = \underset{\subset \scrA}{\scrC} \simeq \underset{\subset \scrB}{\scrD} = \langle d_1, \dots, d_n \rangle_\scrB$, mit $F: \scrC \rightarrow \scrD$ Isomorphismus und $\scrA, \scrB \models \dlo$.
		
		OBdA w"ahle $c_1<c_2<\dots<c_n \underset{F}{\mapsto} d_1<d_2<\dots<d_n$. $\longrightarrow F$ in Back-\&-Forth-System.
		\begin{align*}
			&\text{1. Fall: } &a< c_1 \rightarrow &\text{ w"ahle } b<d_1 \text{ in } \scrB \text{, weil } d_i \text{ kein Randpunkt ist.}\\
			&\text{2. Fall: } &a>c_n \rightarrow &\text{ w"ahle } b<c_n \text{ in } \scrB \text{, weil } d_i \text{ kein Randpunkt ist.}\\
			&\text{3. Fall: } &\exists i \mid c_i<a<c_{i+1} \rightarrow &\text{ w"ahle } b \text{ zwischen } d_i \text{ und } d_{i+1} \text{ weil } \scrB \text{ dicht ist.}
		\end{align*}
		
		Vollst"andigkeit folgt, weil Unterstruktur und Punkt zu Punkt.
	\end{enumerate}
\end{example}

%SITZUNG 5
\begin{example}[Vektorraum]
	Sei $K$ ein K"orper, $\scrL_\text{VR} = \{0, +, f_\lambda \}_{\lambda \in K}$. Dann ist die Theorie $\underset{\substack{\rotatebox{90}{$=$} \\ \text{ unendliche}\\ K\text{-VR}}}{T} = \{\Forall x \Forall y \Forall z \dots \} \dots$\footnote{diese Theorie ist axiomatisierbar, f"ur eine beispielhafte Axiomatisierung vergleiche Klausur zu mathematische Logik im SS 2019.} vollst"andig und hat Quantorenelimination.
	
	Wie zuvor gibt es zwei verschiedene Methoden, um Quantorenelimination zu zeigen:
	\begin{enumerate}
		\item Betrachte die folgende primitive Existenzformel: 
		\begin{equation*}
			\varphi[x_1, \dots, x_n] = \Exists y (\bigwedge\limits_{\text{endlich}} (\lambda_1 x_1 + \dots + \lambda_n x_n + \lambda_y \dot= 0) \land \bigwedge\limits_{\text{endlich}} \lnot (\mu_1 x_1 + \dots + \mu_n x_n \dot= 0)
		\end{equation*}
		Jetzt gibt es zwei M"oglichkeiten:
		\begin{enumerate}
			\item \emph{Alle $\lambda$ vor der Variable $y$ sind Null} $\rightarrow \underbrace{\bigwedge\limits_{\text{endlich}} \lambda x_1 + \dots + \lambda_n x_n = 0}_{\psi[x_1 \dots x_n]}$
			
			\item \emph{Es gibt ein $\lambda \neq 0$}. Dann gilt OBdA: $y \dot= \lambda_1 x_1 + \dots + \lambda_n x_n$. Ersetze jetzt jedes Vorkommen von y durch $\tilde{\lambda}_1 x_1 + \dots + \tilde{\lambda}_n x_n$. Erhalte eine quantorenfreie Bedingung in $x_1, \dots x_n$.
		\end{enumerate}
		
		\item (semantisch)
		
		\underline{Ansatz:} \begin{align*}
			&\setQ &&? &\setQ \oplus \setQ \\
			&\langle 2 \rangle &&\simeq &\langle (3,7) \rangle
		\end{align*}
		Wir brauchen also: $\scrA$ und $\scrB$ undendlichdimensional, um ein Back \& Forth-System zu konstruieren. Es sei dazu
		\begin{equation*}
			\tilde{\scrA} \succeq \scrA \supset \langle c_1, \dots, c_n \rangle \simeq \langle d_1, \dots , d_n \rangle \subset \scrB \preceq \tilde{\scrB}
		\end{equation*}
		f"ur $\tilde{\scrA}, \tilde{\scrB}$ undendlichdimensional.
		
		Insbesondere gilt jetzt auch: 
		\begin{equation*}
			\scrA \models \varphi[c_1, \dots, c_n] \Leftrightarrow \tilde{\scrA} \models \varphi[c_1, \dots, c_n]
		\end{equation*}
		
		Angenommen $\langle c_1, \dots, c_n \rangle \overset{F}{\longrightarrow} \langle d_1, \dots, d_n \rangle$ liegt in einem Back \& Forth-System zwischen $\tilde{\scrA}$ und $\tilde{\scrB}$. Dann folgt insbesondere auch:
		\begin{equation*}
			\tilde{\scrB} \models \varphi[d_1, \dots, d_n] \Leftrightarrow \scrB \models \varphi[d_1, \dots, d_n]
		\end{equation*}
		Es ergeben sich also die folgenden beiden Fragen:
		\begin{enumerate}
			\item \underline{Finden wir ein Back \& Forth-System zwischen $\tilde{\scrA}$ und $\tilde{\scrB}$?} 
			
			Angenommen also wir haben $\tilde{\scrA}$ und $\tilde{\scrB}$ bereits konstruiert. Zeige: Es gibt ein Back \& Forth-System.
			%TODO Grafik einf"ugen
			\begin{enumerate}
				\item[$c \in$ UR:] trivial.
				\item[$c \notin$ UR:] $\dim_K \tilde{\scrB} = \infty \geq n+1 \longrightarrow$ es gibt ein $d \notin \langle d_1, \dots, d_n \rangle \Rightarrow~G$ die Erweiterung \begin{align*}
					\langle c_1, \dots, c_n \rangle &\longrightarrow \langle d_1, \dots, d_n \rangle\\c_i &\longmapsto d_i\\c &\longmapsto d
				\end{align*}
			\end{enumerate}
		
			\item \underline{Zur Existenz von $\tilde{\scrA}, \tilde{\scrB}$:}
			
			So funktioniert es nicht: $\vdig(\scrA) \cup \{\Exists x \Exists y \lnot (\lambda x + \mu y \dot{+} 0)\}_{\substack{\lambda, \mu \in K\\(\lambda, \mu) \neq (0,0)}}$.
			
			Seien $(e_i)_{i \in \setN}$ neue Konstantenzeichen.
			\begin{equation*}
				\underbrace{\vdig(\scrA) \cup \{\lnot \sum_i \lambda_i e_i \dot=0\}_{\substack{(\lambda_1, \dots, \lambda_n) \in K^n \setminus \{(0, \dots, 0) \}\\n \in \setN}}}_\text{endlich konsistent}
			\end{equation*}
		\end{enumerate}
		Zur Vollst"andigkeit: Das endliche Erzeugnis zweier nicht-trivialer Vektoren ist Isomorph, somit folgt Vollst"andigkeit.
	\end{enumerate}
\end{example}

\begin{example}{ACF}\label{acf:ex}
	Wir betrachten jetzt die Theorie algebraisch abgeschlossener K"orper (ACF) in der Ringsprache $\scrL_\text{Ring}=\{0, 1, +, -, \cdot \}$.
	\begin{equation*}
		\acf = \begin{cases}
		\text{K"orperaxiome}\\
		\{\Forall x_0 \Forall x_1 \dots \Forall x_{k-1} \Exists y (y^k + x_{k-1} y^{k-1} + \dots + x_1 y + x_0 \dot= 0) \}_{k \geq 1}
		\end{cases}
	\end{equation*}
	$\acf$ hat Quantorenelimination, ist aber nicht vollst"andig. Die Vervollst"andigungen sind $\underbrace{\acf_0}_{1+1+ \dots + 1 \dot= 0}$ und $\underbrace{\acf_p}_{\underbrace{1+\dots+1}_{p\text{-Mal}} \dot= 0}$ f"ur jede Primzahl $p$.
\end{example}

\begin{theorem}[Kurzeinf"uhrung Galois'sche Theorie]
	%TODO hier noch die Einf"uhrung abschreiben
\end{theorem}

\begin{proof}[Beweis \ref{acf:ex}]
	Betrachte OBdA die Abbildung
	\begin{equation*}
		F = \quo (\langle c_1, \dots, c_n \rangle) \longrightarrow \quo (\langle d_1, \dots, d_n \rangle)
	\end{equation*}
	\underline{Fall 1:} $a$ ist algebraisch "uber $K$
	
	$\hookrightarrow$ sei $m_a(T)$ das Minimalpolynom von $a$ "uber $K$. $F(m_a)(T)$ ist ein normiertes Polynom "uber $\quo(\langle d_1, \dots, d_n \rangle) \subset B$.
	
	$B$ ist algebraisch abgeschlossen $\Rightarrow$ es gibt $b$ in $B$ mit $F(m_a)(b)=0 \overset{\text{Galoistheorie}}{\Longrightarrow} F$ l"asst sich erweitern. 
	
	\underline{Fall2:} $a$ ist transzendent "uber $K = \quo (\langle c_1, \dots, c_n \rangle)$.
	
	Wenn wir ein $b \in B$ finden, welches transzendent "uber $\quo(\langle d_1, \dots, d_n \rangle)$ ist
	
	$\hookrightarrow \ring_A (K, a) \simeq \ring_B (F(K), b)$
	
	\underline{Ziel:} Wir brauchen $\scrA \preceq \tilde{\scrA}$ mit unendlich vielen Elementen, welche algebraisch unabh"angig sind.
	\begin{equation*}
		\underbrace{\vdig(A) \cup \{\lnot (B(e_1, \dots, e_n)\dot= 0) \}_{\substack{P \in A[T_1, \dots, T_n] \setminus \{0\}\\P(e_1, \dots e_n) \neq 0}}}_{\text{endlich konsistent}}
	\end{equation*}
\end{proof}


%Sitzung 6
\section{Ultrafilter \& der Satz von Ax}
\underline{Anwendung:} Wir wollen eine Aussage der folgenden Art bekommen: Sei $f: \substack{\setC \longrightarrow \setC \\z \longmapsto z^2}$.\\
$\rightarrow f$ ist surjektiv, aber nicht injektiv.
 
\begin{theorem}[Ax]\label{ax:thm}
	Sei $f: \underset{z \longmapsto z^2}{\setC^n \longrightarrow \setC^n}$ eine polynomiale\footnote{polynomial bedeutet, dass jede Koordinate der Abbildung durch Polynome gegeben ist.} injektive Abbildung. Dann ist $f$ surjektiv.
\end{theorem}

\underline{Motivation:} Sei $G$ eine Gruppe der Ordnung $p$. F"ur einen K"orper der Charakteristik $p$ bekommen wir dann:
\begin{align*}
	\underbrace{\setZ / p\setZ}_{\ni \bar{g}} \underset{\text{wirkt}}{\curvearrowright} \underbrace{K}_{\substack{\text{K"orper der}\\ \text{Charakteristik}\\p}} &\longrightarrow K\\
	x &\longmapsto \underbrace{1+ \cdots +1}_{g \text{-Mal}}+ x\\
	\rightarrow h+(g+x) = (h+g)+x
\end{align*}
F"ur einen K"orper der Charakteristik $0$:
\begin{align*}
	&\underbrace{\setZ/p \setZ}_{\ni \bar{k}} \underset{\text{wirkt}}{\curvearrowright} &&\setC \longrightarrow \setC\\
	&\underbrace{\mu_p}_{\substack{p\text{-te Einheits-}\\ \text{wurzel in } \setC}} = \{e^{\frac{2\pi i k}{p}} \}_{0 \leq k < p} &&z \longmapsto \omega z\\
	&\rightarrow \omega_1 (\omega \cdot z) = (\omega_1 \omega)\cdot z
\end{align*}

\begin{theorem}[Lefschetz'sches Prinzip]\label{lef:thm}
	Eine Aussage $\chi$ in der Ringsprache $\scrL_\text{Ring}$ gilt f"ur $\setC$ genau dann, wenn es unendlich viele Primzahlen $p$ derart gibt, dass $\chi$ in einem algebraisch abgeschlossenen K"orper der Charakteristik $p$ gilt.
\end{theorem}

\begin{proof}[Beweis von Satz \ref{ax:thm} (Satz von Ax)]
	Sei $f: \setC^n \longrightarrow \setC^n$ injektiv. Die Aussage "`$f$ injektiv $\Rightarrow f$ surjektiv"' l"asst sich als $\scrL_\text{Ring}$-Aussage schreiben.
	
	D. h. es gen"ugt zu zeigen, dass diese Aussage f"ur \underline{alle} K"orper $\mathbb{F}^\text{alg}_p$ gilt.
	
	\underline{Was ist $\mathbb{F}^\text{alg}_p$?}\marginpar{Galoistheo.} Ein algebraischer abgeschlossener K"orper der Charakteristik $p$.
	\begin{align*}
		\mathbb{F}^\text{alg}_p = \bigcup\limits_{n \in \setN} F_n \text{, wobei } F_n &\subset F_{n+1} \text{ endliche K"orper mit Charakteristik } p.\\
		F_1&=\{0,1\}\\
		F_2&= \cdots\\
		&\vdots
	\end{align*}
	Sei nun $g: (\mathbb{F}^\text{alg}_p)^n \longrightarrow (\mathbb{F}^\text{alg}_p)^n$ eine surjektive polynomiale Abbildung.
	
	\underline{Zeige: $g$ ist surjektiv.} Sei $(b_1, \dots, b_n) \in (\mathbb{F}^\text{alg}_p)^n$. Dann gibt es ein $N$, sodass $b_i \in \mathbb{F}_n$ f"ur $\mathbb{F}_n$ endlich.
	
	Ferner k"onnen wir $N$ so w"ahlen, dass alle Koeffizienten aus $g$ in $\mathbb{F}_n$ liegen.
	\begin{align*}
		g_{\upharpoonright \mathbb{F}^n_N}: \underbrace{\mathbb{F}^n_N}_{\text{endlich}} \longrightarrow \underbrace{\mathbb{F}^n_N}_{\text{endlich}} \text{ ist }&\text{injektiv (geerbt)}\\
		&\Downarrow \text{ endlich} \\
		&\text{surjektiv}
	\end{align*}
\end{proof}

\begin{proof}[Beweis vom Prinzip von Lefschetz (Satz \ref{lef:thm})]
	\begin{itemize}
		\item["`$\Rightarrow$"'] Sei $\chi$ eine $\scrL_\text{Ring}$-Aussage derart, dass $\setC \models \chi$. Dann ist $\underbrace{\acf_0}_{\substack{\text{alle elementar}\\ \text{"aquivalent}}} \cup \{\lnot \chi \}$ inkonsistent, weil $\acf_0$ vollst"andig ist.
		
		Dann gibt es eine endliche Teilmenge $T_0 \subset \acf_0 \cup \{\lnot \chi \}$, welche inkonsistent ist. $\Rightarrow$ Es gibt ein $N \in \setN$ sodass:
		\begin{equation*}
			T_0 \subset \underbrace{\acf \cup \{\lnot (\underbrace{1+ \cdots +1}_k \dot= 0) \}_{k<N} \cup \{\lnot \chi \}}_{\text{inkonsistent}}
		\end{equation*}
		F"ur $p>N$ eine Primzahl: $\acf_p \models \chi$
		
		\item["`$\Leftarrow$"'] $\rightsquigarrow$ Ultrafilter und \nameref{los:thm}
	\end{itemize}
\end{proof}
		
	\heading{Exkurs:} Sei im Folgenden $I \ne \emptyset$.
	\begin{definition}
		Ein Ultrafilter $\scrU$ auf $I$ ist ein endlich additives Wahrscheinlichkeitsma"s
		\begin{equation*}
			\mu_\scrU : \scrP(I) \longrightarrow \{0,1\}
		\end{equation*}
	\end{definition}
	\begin{remark}
		Die Definition entspricht der von Blatt 1 Aufgabe 3, denn:
		\begin{enumerate}
			\item $\mu_\scrU(I)=1$, $\mu_\scrU(\emptyset)=0$.
			\item $\underset{X\subset Y \subset I}{\mu_\scrU(X)}=1 \Rightarrow \mu_\scrU(Y)=1$
			\item Angenommen $\mu_\scrU(X)=\mu_\scrU(Y) = 1$ aber $\mu_\scrU(X \cap Y) = 0$. Dann gilt $X = X \setminus Y \dot\cup X \cap Y \Rightarrow \mu_\scrU(X \setminus Y) = 1$ und $\mu_\scrU(Y \setminus X) =1$, sowie $I \supset X \cup Y = X \setminus Y \dot\cup Y\setminus X \dot\cup X \cap Y$. $\rightsquigarrow \mu_\scrU(I)= 1 \geq 1+1+0$, ein Widerspruch.
			\item Gegeben $X \subset I$ entweder $\underset{\mu_\scrU(X)=1}{X \in \scrU}$ oder $\underset{\mu_\scrU(I\setminus X)=1}{I\setminus X \in \scrU}$
		\end{enumerate}
	\end{remark}
	
	\begin{definition}
		Ein Hauptultrafilter ist ein Ma"s der Form $\delta_x$ f"ur ein $x \in I$.
	\end{definition}

	\begin{definition}
		Falls I undendlich ist, so gibt es generisch/reiche Ultrafilter, n"amlich die Ultrafilter, welche alle koendlichen Mengen enthalten.
	\end{definition}

	\begin{definition}
		Angenommen $(\scrA_i)_{i \in I}$ ist eine $\scrL$-Struktur. Sei ferner $\scrU$ ein Ultrafilter. Definiere eine "Aquivalenzrelation\footnote{vergleiche dazu Blatt 1, Aufgabe 3} auf $\underbrace{\prod\limits_{\scrU}}_{\substack{\text{karthesisches}\\ \text{Produkt}}} A_i$:
		\begin{equation*}
			(a_i)_{i \in I} \sim_\scrU (b_i)_{i \in I} \Longleftrightarrow \{i \in I \mid a_i = b_i \} \in \scrU \Longleftrightarrow \mu_\scrU(\{i \in I \mid a_i = b_i \})=1
		\end{equation*}
	\end{definition}
	\begin{definition}
		Sei $\underbrace{\prod\limits_\scrU}_{\ne \emptyset} A_i$ die Menge $\prod\limits_{i \in I} A_i / \sim_\scrU$. Wir definieren Interpretationen der Symbole aus $\scrL$ auf $\prod\limits_{\scrU} A_i$:
		\begin{itemize}
			\item Sei $c \in \scrL$ ein Konstantenzeichen. Definiere:
			\begin{equation*}
				c^{\prod\limits_{\scrU} A_i} = (c^{\scrA_i})_{i \in I} / \sim_\scrU
			\end{equation*}
			\item Sei $f \in \scrL$ ein $n$-stelliges Funktionszeichen. Definiere:
			\begin{equation*}
				f^{\prod\limits_{\scrU} A_i} ([a_1]_\scrU, \dots, [a_n]_\scrU)= (f^{\scrA_i}(a_1^i, \dots, a_n^i))_{i \in I} / \sim_\scrU 
			\end{equation*}
			Ist das wohldefiniert? Ja, denn fast "uberall gleich.
			\item Sei $\mathcal{R}$ ein $m$-stelliges Relationszeichen auf $\scrL$. Definiere:
			\begin{equation*}
				([a_1]_\scrU, \dots, [a_m]_\scrU) \in \mathcal{R}^{\prod\limits_{\scrU} A_i} \Longleftrightarrow \{i \in I \mid (a_1^i, \dots, a_n^i) \in \mathcal{R}^{\scrA_i} \} \in \scrU
			\end{equation*}
		\end{itemize}
	\end{definition}
	Wenn $\scrU$ ein Hauptfilter ist, dann ist er erzeugt vom Element $\{i_0 \}$.
	\begin{align*}
		\overbrace{\prod\limits_{\scrU} \scrA_i}^{\scrL \text{-Struktur}} &\overset{\varphi}{\longrightarrow} \scrA_{i_0} \text{ ist ein Isomorphismus}\\
		(a_i)_{i \in I} / \sim_\scrU &\longmapsto a_{i_0}
	\end{align*}
	
	\begin{definition}
		Wenn $\scrA$ eine $\scrL$-Struktur und $\scrU$ ein Ultrafilter ist, dann ist $\scrA^\scrU = \prod\limits_{\scrU} \scrA$ die Ultrapotenz.
	\end{definition}

	\begin{example}
		Sei $\scrU$ ein reicher/generischer Ultrafilter auf $\setN$. Betrachte $\mathcal{N} = (\setN, <)$.
		\begin{equation*}
			\mathcal{N}^\scrU \ni {(1,2,3, \dots)/\sim_\scrU} > {(1,1,1, \dots)/ \sim_\scrU}
		\end{equation*}
	\end{example}

%Sitzung 7
\begin{theorem}[Satz von Łoś]\label{los:thm}
	Sei $\scrU$ ein Ultrafilter auf $I$, $(\scrA_i)_{i \in I}$ eine Familie von $\scrL$-Strukturen, $\varphi[x_1, \dots, x_n]$ eine $\scrL$-Formel und $[a_1]_\scrU , \dots, [a_n]_\scrU \in \prod\limits_{\scrU} A_i$. Dann gilt:
	\begin{equation*}
		\prod\limits_{\scrU} \scrA_i \models \varphi[[a_1]_\scrU , \dots , [a_n]_\scrU] \Longleftrightarrow \{i \in I \mid \scrA_i \models \varphi[a^1, \dots, a^n] \} \in \scrU
	\end{equation*}
\end{theorem}
\begin{proof}
	Induktiv "uber den Aufbau von $\varphi$. Sei $\varphi = (t_1 \dot= t_2)$. Dann gilt:
	\begin{align*}
		\prod\limits_{\scrU} A_i &\models (t_1[[a_1]_\scrU , \dots, [a_n]_\scrU] \dot= t_2[[a_1]_scrU, \dots, [a_n]_\scrU])\\
		&\Leftrightarrow t_1^{\prod\limits_{\scrU} \scrA_i}[[a_1]_\scrU , \dots, [a_n]_\scrU] \dot= t_2^{\prod\limits_{\scrU} \scrA_i}[[a_1]_\scrU, \dots, [a_n]_\scrU]\\
		&\underset{\substack{\text{induktiv "uber}\\ \text{den Aufbau}}}{\Leftrightarrow} \{i \in I \mid \scrA_i \models t_1[a_i^1, \dots, a_i^n]\dot= t_2 [a_i^1, \dots, a_i^n] \} \in \scrU
	\end{align*}
\end{proof}

\begin{corollary}
	Sei $\scrA$ eine $\scrL$-Struktur und $\scrU$ ein Ultrafilter auf $I$. Betrachte $\scrA^\scrU = \prod\limits_{\scrU} \scrA$. Das ist eine elementare Erweiterung von $\scrA$ bez"uglich der Abbildung \marginpar{Einbettung,\\injektiv}$\underset{a \longmapsto (a)_{i \in I}/\sim_\scrU}{A \longrightarrow \prod\limits_{\scrU} A}$.
\end{corollary}
\begin{proof}
	Sei $\varphi$ eine $\scrL$-Formel, $a_1, \dots, a_n \in A$. Zu zeigen ist:
	\begin{equation*}
		\scrA \models \varphi[a_1, \dots, a_n] \Longleftrightarrow \scrA_i^\scrU \models \varphi[[a_1]_\scrU , \dots, [a_n]_\scrU]
	\end{equation*}
	\underline{"`$\Rightarrow$"':} Mit \nameref{los:thm} gilt:
	\begin{equation*}
		\scrA_i^\scrU \models \varphi[[a_1]_\scrU , \dots, [a_n]_\scrU] \Longleftrightarrow \{i \in I \mid \scrA \models \varphi[a_1, \dots, a_n] \} \in \scrU
	\end{equation*}
	Da dieser Ausdruck jedoch der gesamten Menge I entspricht, folgt die Behauptung direkt.
	
	\underline{"`$\Leftarrow$"':} Die leere Menge liegt nicht in $\scrU$, also gibt es $i$ sodass die Formel gilt, da diese jedoch von $i$ unabh"angig ist, gilt sie immer.
\end{proof}

\begin{proof}[Beweis \nameref{lef:thm} \eqref{lef:thm} "`$\Leftarrow$"']
	Sei
	\begin{align*}
		S= \left\{p \text{ Primzahl}\mid 
		\begin{array}{l l}
		\text{ein algebraisch abgeschlossener K"orper mit}\\ 
		\text{Charakteristik } p\text{ erf"ullt die Aussage } \chi  \end{array}\right\}
	\end{align*}
	
	Zeige: $S$ ist unendlich. Sei $P \subset \setN$ Primzahlen. Betrachte jetzt
	\begin{equation}
		\scrB = \{X \cap S \subset P \mid X \subset P \text{ koendlich} \}
	\end{equation}
	Ist $\scrB$ eine Filterbasis? $X \cap S = \emptyset$ ist endlich $\Longleftrightarrow~ S \subset P \setminus X$ unendlich, ein Widerspruch.
	
	Weiter gilt $(X_1 \cap S) \cap (X_2 \cap S) = \underbrace{(X_1 \cap X_2)}_\text{koendlich} \cap S$.
	
	$\overset{\text{Blatt 1}}{\Rightarrow}$ es gibt einen Ultrafilter, welcher alle Elemente aus $\scrB$ enth"alt.
	
	Sei im Weiteren $\scrU$ ein Ultrafilter auf $P$, welcher $\scrB$ enth"alt. $X \cap S \in \scrU$ ist f"ur alle $X \subset P$ koendlich.
	
	$\hookrightarrow~ \scrU$ ist reich (kein Hauptultrafilter). F"ur $p_0 \in P$ ist $P \setminus \{p_0\}$ koendlich. $\Rightarrow~P \setminus \{p_0\} \cap S \in \scrU$.
	
	$\hookrightarrow~ S \in \scrU$
	
	Sei $K= \prod\limits_{\scrU} K_p$, wobei $K_p$ ein algebraisch abgeschlossener K"orper der Charakteristik $p$ ist derart, dass
	
	\begin{equation*}
		\begin{cases}
			K_p \models \chi &p \in S\\
			\text{egal }_{\text{bspw. } \mathbb{F}_p^{\text{alg}}} &p \notin S
		\end{cases}	
	\end{equation*}
	\begin{enumerate}
		\item $K \models \acf_0$
		\item $K \models \chi$, weil $\{p \in P \mid K_p \models \chi \} \supset S \in \scrU$
	\end{enumerate}
	$\acf_0$ ist vollst"andig $\Rightarrow~ \setC \models \chi$.
\end{proof}

\begin{theorem}[Kompaktheitssatz]\label{kmp:thm}
	Eine Theorie $T$ ist genau dann konsistent, wenn sie endlich konsistent ist.
\end{theorem}
\begin{proof}
	OBdA ist $T$ unendlich. Sei $I = \{\emptyset \neq S \subset T \text{ endlich} \}$. F"ur $s \in I$ gibt es eine $\scrL$-Struktur $\scrA_s$, sodass $\scrA_s \models \chi$ f"ur jedes $\chi \in s$. Sei weiter \begin{equation*}
		B_s = \{t \in I \mid \scrA_t \models \chi \text{ f"ur jedes } \chi \in s \}
	\end{equation*}
	Ist $\scrB = \{B_s\}_{s\in I}$ eine Filterbasis?
	\begin{enumerate}
		\item $\emptyset \neq B_s \ni s$
		\item $B_{s_1} \cap B_{s_2} = \{t \in I \mid \scrA_t \models \chi \text{ f"ur alle } \chi \text{ aus } s_2 \} = B_{s_1 \cup s_2} \in \scrB$!
	\end{enumerate}
	Sei $\scrU$ ein Ultrafilter auf $I$, sodass $B_s \in \scrU$ f"ur jedes $\emptyset \neq s \subset T$ endlich. Sei $\scrA = \prod\limits_{\scrU} \scrA_s$.
	
	Zu zeigen ist: $\scrA \models T$ (sei $\chi \in T$, zeige $\scrA \models \chi$).
	
	$\overset{\ref{los:thm}}{\Longleftrightarrow}~ \underbrace{\{s \in T \mid \scrA_s \models \chi \}}_{B_{\{\chi\}}} \in \scrU$
\end{proof}

\part{Typen und Saturation}
\section{Typen}
Sei im Folgenden $\scrL$ eine Sprache und $\scrA$ eine $\scrL$-Struktur.
\begin{definition}
	Ein partieller Typ $\sum (x_1, \dots, x_n)$ mit Parametern aus $B$ ist eine Kollektion von Formeln in der Sprache $\scrL \cup \{b\}_{b \in B}$, welche in der (kanonischen) $\scrL \cup \{b\}_{b \in B}$-Struktur $\scrA$ endlich erf"ullbar ist, das hei"st f"ur alle $\varphi_1, \dots, \varphi_m \in \sum$ gibt es ein Tupel $(a_1, \dots, a_n) \in A^n$ mit $\scrA \models \varphi_i(a_1, \dots, a_n)$ f"ur $1 \leq i \leq m$.
	
	$\scrA$ realisiert $\sum$, falls es ein Tupel $(a_1, \dots, a_n)$ gibt, sodass $\scrA \models \varphi[a_1, \dots, a_n]$ f"ur alle $\varphi \in \sum$. Sonst $\begin{array}{c}
		\text{vermeidet}\\ 
		\text{"ubergeht}
	\end{array} \scrA$ den partiellen Typ $\sum$. 
\end{definition}
\begin{example}
	Betrachte $(\setR, 0, <)$. Sei $\sum (x) = \{0 < x<q \}_{\substack{q \in \setQ \\ q>0}}$ ein partieller Typ.
	
	Wird $\sum$ realisiert oder vermieden? $\rightsquigarrow$ vermieden
	
	Sei jedoch $\sum' = \{\sqrt{2} \leq x < q \}_{\substack{q \in \setQ \\ q > \sqrt{2}}}$. $\rightsquigarrow$ realisiert von $\sqrt{2}$
	
	Betrachte nun $\sum$ auf $\prod\limits_{\scrU} \setR$. Hier realisiert $(1, \frac{1}{2}, \frac{1}{3}, \frac{1}{4}, \dots)$ den partiellen Typen $\sum$!
\end{example}
\begin{remark}
	Sei $\scrA$ eine unendliche Struktur. Dann gibt es immer einen partiellen Typen, der vermieden wird: $\{\lnot (x \dot= a) \}_{a \in A}$.
\end{remark}
\end{document}